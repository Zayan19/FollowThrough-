\documentclass{article}
\usepackage[utf8]{inputenc}

\title{Statement of Problem Revision 1}
\author{
  Djordje Petrovic\\ \texttt{1306258}\\ \texttt{petrod@mcmaster.ca} \and George Plukov\\ \texttt{1316246}\\ \texttt{plukovga@mcmaster.ca} \and Zayan Imtiaz\\ \texttt{1152665}\\ \texttt{imtiaz3@mcmaster.ca} \and Philip Habib \\ \texttt{1310455}\\ \texttt{habibp@mcmaster.ca}
}
\date{April 2017}

\begin{document}

\maketitle
\newpage

\section{Introduction}
When practising alone on the basketball court players should be able to receive real-time feedback so that they know how to improve their skills. In most situations if a player is looking for somebody to assist them in perfecting their form, they will bring another person onto the court with them so that they may assess the player and provide suggestions as to how to fix any problems in their technique. Obviously there is a major question of unavailability of the people you can have advising you. What if your person of choice has no time during the periods when you choose to train basketball?

\section{The Solution}
The solution for the proposed problem is quite a simple one; what if there was a tool that could give you real time diagnostic results whenever you plan to practice? FollowThrough is the tool to do exactly this. The device would be optimally set up in a position where it can track your movements and how they correspond to the quality of your shots. It would examine the arch of your shots and depending on the location you are throwing the ball will tell you how much higher or lower it needs to be or how much more or less force you need to put to the shot.
\\ \\
FollowThrough would only require a simple camera to complete most of its functions. First it is to keep track of the score, and analyse both shot form and ball arch. It would track the shots and analyse them in a real time setting which means it will give live feedback. Following that the tool 
will upload the information to the cloud where it can be stored for later review. FollowThrough is a simple to use tool that can guide the next generation of basketball players to perform better and ultimately become famous athletes.

\section{Challenges}
The most foreseeable challenge in this case is that of effectively tracking the ball. Simply relying on colour is not feasible due to the fact that if there is anything similarly coloured in the background the tracking gets thrown off. A slightly less foreseeable issue would be the efficacy of the algorithm we use to track the ball. If the algorithm is too slow, the user response to the product will be quite negative.

\section{Objectives}
The obvious objective here is to build a fully functioning virtual basketball coach by the end of the school year. Anything further is circumstantial depending on how well the project goes. If the product is usable and has very few bugs, it is possible that we will actually move the product to shipping.

\section{Assumptions}
The biggest assumption we are making is that the user will be at least slightly competent with modern day technology. This will allow them to seamlessly use the product. On the software side we are assuming that we will find a good alternative to colour tracking. On the hardware side we are assuming that the hardware will be easy to connect and easy to obtain.

\section{Constraints}
The major constraint in this case is time. All components of our project must be complete within the course of eight months. This constraint is quite tantalising as motion tracking is still a relatively new concept, so figuring the ins and outs of the software may prove to be tasking under the pressure of time.

\end{document}

%*******************************************************
% Project Info
%*******************************************************
\pdfbookmark[1]{Project Information}{Project Information}
\chapter{Project Information}

\section{Preface}
This Requirements Analysis is done following the Volere template.

% Background of the project
\section{Background of the project}
\hspace*{4mm}FollowThrough is to be a multi-platform system which can provide basketball players with real-time feedback on how to improve their shooting form. The hardware/software involved is to film the user while practicing, processing the information on the spot and returning usable advice which can change the player’s game for the better. The information is afterwards sent to the cloud where users may review their advice at home. It will also be designed to keep track of score for the user’s convenience.

Currently there is no affordable method for an everyday player to have professional assistance in improving their shot form. Follow through hopes to be provide basketball player with a cheap and accurate alternative to expensive coaching and training camps.

Ideally the project will be able to function on any laptop camera or webcam.

% Motivation 
\section{Motivation}
\hspace*{4mm}When practicing alone on the basketball court, players should be able to receive real-time feedback so that they know how to improve their skills. In most situations if a player is looking for somebody to assist them in perfecting their form, they will bring another person with them so that they may assess the player and provide suggestions as to how to fix any problems in their technique. Obviously this is not an optimal solution. Not everyone is able to have someone advising them at all times. Coaches are a possibility but can be extremely expensive and availability again becomes an issue.


% Challenges
\section{Challenges}
\begin{itemize}
    \item Ensuring FollowThrough runs efficiently with any modern laptop camera.
    \item Ensuring similar colours in the background do not hinder tracking.
    \item Ensuring user's manual is user friendly.
    \item Ensuring the average user can setup and successfully use FollowThrough.
\end{itemize}

% Goal
\section{Goal}
\hspace*{4mm}The aim of FollowThrough is to eliminate the need for shot assistance while on the basketball court. FollowThrough will be software that works with any camera and laptop to provide real time advice and feedback to basketball players right on the court. FollowThrough will also save the analysis of the shot to the cloud for later review.

% Relevant Facts and Assumptions
\section{Relevant Facts and Assumptions}
\subsection{Facts}
\begin{itemize}
    \item A camera is required to use this software.
    \item The hardware associated with counting score is optional for the user.
\end{itemize}

\subsection{Assumptions}
\begin{itemize}
    \item Users have a rudimentary knowledge of how to operate a computer.
    \item Users have a camera built into their device or an external usb web-cam.
    \item Users are able to follow instructions to setup and calibrate the device.
\end{itemize}

% Constraints
\section{Constraints}
\subsection{Mandated Constraints}
\textbf{Description:} FollowThrough is to be very easy to set up. Setting up the physical portions of the product will be easy because of well thought out instructions to avoid confusion.\\
\textbf{Rationale:} The product has to be easy to use.\\
\textbf{Fit Criterion:} The product is to have minimal requirements for set-up.\\\\
%%%%%%
\textbf{Description:} FollowThrough will provide both auditory and visual feedback to the user.\\
\textbf{Rationale:} This will increase the accessibility of the product and open it up to for use to people with disabilities and even further assist both auditory and visual learners.\\
\textbf{Fit Criterion:} The output will be extensively tested and approved before release.\\\\
%%%%%%
\textbf{Description:} The product will be able to be used on a variety of operating systems such as Windows 7, 8, 10, Mac OSX …etc.\\
\textbf{Rationale:} The product being cross-platform will make it easier for a large market of people to access.\\
\textbf{Fit Criterion:} The product will be tested on various platforms to ensure compatibility.\\\\
%%%%%%
\textbf{Description:} The product will upload all the gathered information in a readable form to a cloud database and the user may access this later to review their progress.\\
\textbf{Rationale:} This will allow users to analyze and utilize the data tracked.\\
\textbf{Fit Criterion:} A page will be developed and tested to ensure it works fluidly.
%%%%%%
\subsection{Schedule Constraints}
\begin{itemize}
    \item Proof of Concept Demonstration November $21^{st} – 25^{th}$ 2016.
    \item Demonstration Revision 0 February $13^{th} – 17^{th}$ 2017.
    \item Final Demonstration Exam Period 2017.
\end{itemize}

\subsection{Budget Constraints}
\begin{itemize}
    \item Budget constraints can be seen below in the cost/benefit analysis
\end{itemize}

% Naming Conventions and Definitions
\section{Naming Conventions and Definitions}
\subsection{Definition of all Terms}
\begin{itemize}
    \item \textbf{FollowThrough:} The name of the product at hand.
    \item \textbf{Arduino:} A simple micro-processing unit that can be programmed to complete various tasks.
    \item \textbf{OpenCV:} An open source computer vision and machine learning software library.
\end{itemize}


